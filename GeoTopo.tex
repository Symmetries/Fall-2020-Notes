\documentclass[12pt]{book} % non stream mode
%\documentclass[17pt]{extbook} % stream mode
\usepackage[utf8]{inputenc}

\usepackage{import}

\import{./template/}{defaults}
\import{./template/}{common}
\import{./template/}{notes}

\newcommand{\T}{\mathcal T}
\newcommand{\B}{\mathcal B}
\DeclareMathOperator{\Int}{Int}
\DeclareMathOperator{\Cl}{Cl}

\title{Geometry and Topology}
\author{Diego Lopez and Gabriela Moisescu-Pareja}
\date{Fall 2020}

\begin{document}

\maketitle

\chapter{Topological Spaces}

\begin{definition}
A \emph{topology} on a set \( X \) is a collection of subsets of \( X \) satisfying
\begin{enumerate}
\item \( \emptyset \in \T \) and \( X \in T \);
\item if \( U_\alpha \in \T \) for all \( U_\alpha \in A \) for some index set \( A \), then \[ \bigcup_{\alpha \in A} U_\alpha \in \T; \]
\item if \( U_1, U_2 \in \T \) then \( U_1 \cap U_2 \in \T \).
\end{enumerate}
A set \( U \in \T \) called an \emph{open set}.
The pair \( (X, \T) \) is called a topological space.
\end{definition}

\begin{example} \skipline
\begin{itemize}
\item The \emph{discrete topology} on a set \( X \) is its powerset, \ie \( \T = \mathcal P(X) \).
\item The \emph{trivial topology} on a set \( X \) is \( \T = \set{\emptyset, X} \).
\end{itemize}
\end{example}

\begin{definition}
A \emph{closed} subset is the complement of an open subset.
\end{definition}

\begin{definition}
A \emph{finite topological space} is a topological space \( X \) with \( \card X \leq \infty \).
\end{definition}

\begin{definition}
Let \( \T_1, \T_2 \) be two topologies on the same set \( X \).
Then, 
\begin{itemize}
\item \( T_1 \) is said to be \emph{finer} than \( T_2 \) if \( T_1 \supset T_2 \); and
\item \( T_1 \) is said to be \emph{coarser} than \( T_2 \) if \( T_1 \subset T_2 \).
\end{itemize}
\end{definition}

\begin{note}
Two topologies may not be comparable.
That is, there can be topologies \( T_1, T_2 \) on the same set \( X \) such that neither one is coarser nor finer than the other.
\end{note}

\begin{definition}
A collection \( \B \) of subsets of \( X \) is said to be a \emph{basis} if
\begin{enumerate}
\item the union of all sets is \( X \), that is \[ X = \bigcup_{B \in \B} B; \]
\item for all \( B_1, B_2 \in \B \) and each \( x \in B_1 \cap B_2 \) we have that there is some \( B_2 \in \B \) such that \( x \in B_3 \subset B_1 \cap B_2 \).
\end{enumerate}
\end{definition}

\begin{note}
The condition (2) is implied by the following stronger condition
\begin{enumerate}
\item[\( 2'. \)] for all \( B_1, B_2 \in \B \) we have that \( B_1 \cap B_2 \in \B \).
\end{enumerate}
\end{note}

\begin{example}
The open rectangles in \( \R^2 \) satisfy condition \( (2') \).
\end{example}

\begin{definition}
A basis \( \B \) is said to be a \emph{basis for the topology \( \T \)} if \( \T \) equals the collection of unions of subcollections of \( \mathcal B \), \ie 
\[ \T \coloneqq \set*{ \bigcup_{B \in \mathcal A} B \given* \mathcal A \subset B}. \]
Equivalently, for every \( x \in U \in \T \) there is some \( B \in \B \) such that \( x \in B \subset U \).
In either case, we say that \( \B \) \emph{generates} \( \T \).
\end{definition}

\begin{example}
A basis for \( \R^n \) are the open balls
\[ B(x, r) \coloneqq \set{y \in \R^n \given d(x, y) < r } \]
for all \( r > 0 \) and all \( x \in \R^n \).
Another basis are the open rectangles.
\end{example}

\begin{definition}
The \emph{profinitie topology} on a group \( G \) is the topology whose basis consists of left cosets of finite index subgroups
\[ \B = \set{gH \given [G : H] < \infty,\ g \in H}. \]
\end{definition}

\begin{note}
The profinite topology is a topology since
\begin{enumerate}
\item if \( [G : H_1] < \infty \) and \( [G : H_2] < \infty \) then \( [G : H_1 \cap H_2] < \infty \);
\item \( g_1 H_2 \cap g_2 H_2 = g_3 (H_1 \cap H_2) \) for some \( g_3 \in G \).
\end{enumerate}
That is, condition (1) for bases holds since \( [G : G] = 1 < \infty \) and condition (2) holds by note (2).
\end{note}

\begin{definition}
A topological space \( (X, \T) \) is said to be \emph{Hausdorff} if for every \( p, q \in X \) with \( p \neq q \) we have that there exist \( U_p, U_q \in \T \) such that \( p \in U_p \), \( q \in U_q \) and \( U_p \cap U_q = \emptyset \).
\end{definition}

\begin{definition}
If \( U \in \T \) and \( x \in U \), we say that \( U \) is a \emph{neighborhood} of \( x \).
\end{definition}

\begin{note}
When is the profinite topology on \( G \) Hausdorff?
What algebraic property much \( G \) have?
This happens exactly when \( G \) is \emph{residually finite}, that is for all \( g \neq 1_G \), there exists a finite quotient \( G \to \overline G \) such that \( \bar g \neq 1_{\overline G} \).
For example, \( \Z^n \) is residually finite.
\end{note}

\begin{definition}
Let \( (X, \T_X) \) be a topological space.
Let \( Y \subset X \) be a subset.
The \emph{subspace topology} \( \T_Y \) on \( Y \) is given by
\[ \T_Y \coloneqq \set{U \cap Y \given U \in \T_X}. \]
\end{definition}

\begin{note}
Open sets of \( Y \) might not be open in \( X \).
For example, consider \( Y = (0, 1] \subset \R \).
On the other hand, if \( Y \subset X \) is open, then every open set of \( \T_Y \) is open in \( \T_X \).
\end{note}

\begin{definition}
Let \( (X, \T_1) \) and \( (X_2, \T_2) \) be topological spaces.
The \emph{product topology} on \( X_1 \cross X_2 \) is generated by the basis
\[ \B = \set{U_1 \cross U_2 \given U_1 \in \T_1,\ U_2 \in \T_2}. \]
\end{definition}

\begin{note}
The product topology can be similarly defined for finite products.
\end{note}

\begin{definition}
Let \( \mathcal S \) be a collection of subsets of \( X \).
Let \[ \B_\mathcal S = \set*{\bigcap_{i \in F} S_i \given* \card F < \infty,\ S_i \in \mathcal S \text{ for all \( i \in F \)}}. \]
Regard \( X \) as an element of \( \B_\mathcal S \) with \( F = \emptyset \).
Then, \( \B_\mathcal S \) is a basis since \( X \in \B_\mathcal S \) and condition \( (2') \) is satisfied.
We call \( \T_\mathcal S = \T_{\B_\mathcal S} \) the \emph{topology generated by \( \mathcal S \)} and \( \mathcal S \) a subbasis for \( \T_\mathcal S \).
\end{definition}

\begin{note}
The topololy \( \T_\mathcal S \) is the coarsest topology containing \( \mathcal S \).
\end{note}

\begin{example}
The product topology \( X_1 \cross X_2 \) has subbasis
\[ \set{U_1 \cross X_2 \given U_1 \in \T_1} \cup \set{X_1 \cross U_2 \given U_2 \in \T_2}. \]
\end{example}

\begin{definition}
Let \( (X, <) \) be a set with an \emph{order relation}.
That is, a relation which satisfies
\begin{enumerate}
\item non-reflexivity, \ie \( x < x \) does not hold for any \( x \in X \);
\item comparability, \ie for every \( x, y \in X \) we have exactly one of \( x = y \), \( x < y \) or \( y < x \).
\item transitivity, \ie if \( x < y \) and \( y < z \) then \( x < z \).
\end{enumerate}
The \emph{order topology} \( T_< \) s generated by the subbasis
\[ \mathcal S = \set{(x, +\infty) \given x \in X} \cup \set{(-\infty, x) \given x \in X} \]
where
\begin{align*}
(x, +\infty) &\coloneqq \set{y \in X \given x < y} \\
(-\infty, x) &\coloneqq \set{y \in X \given y < x}.
\end{align*}
\end{definition}

\begin{remark}
We may be tempted to try to use the subbasis
\[ \mathcal S = \set{(a, b) \given a, b \in X}. \]
However, we would find that it has some problems.
Indeed, in the case where \( X = \set{x, y} \) with \( x < y \), then the set \( (x, y) \) is the empty set.
However, the order topology on \( X \) is the discrete topology, not the trivial topology.
\end{remark}

\begin{example}
The set \( (\R, <) \) is a motivating ordered set.
The order topology is the same as the open ball topology.
Indeed, each open ball \( (a, b) = (-\infty, b) \cap (a, +\infty) \).
\end{example}

\begin{definition}
Let \( (X_1, <_1) \) and \( (X_2, <_2) \) be ordered sets.
We order the product \( (X_1 \cross X_2) \) using the \emph{lexicographic ordering}.
That is, \( (p_1, p_2) < (q_1, q_2) \) if \( p_1 < q_1 \) or \( p_1 = q_1 \) and \( p_2 < q_2 \).
\end{definition}

\begin{example}
Given \( \R \cross \R \), the lexicographic order topology.
This topology is finer than the usual topology.
All open vertical segments are open subsets.
So, all open ball (or open rectangles) in \( (\R^2, \T) \) are open in \( (\R^2, \T_<) \).
However, every non-vertical line segment \( L \) in \( (\R^2, \T_<) \) is discrete using the subspace topology.
This is very different from the subpace topology obtained from the standard topology \( \T_{\mathbb E} \).
\end{example}

\begin{definition}
Let \( A \leq X \) where \( X \) is a topological space.
The \emph{interior} of \( A \) is the union of open sets of \( X \) lying in \( A \).
The \emph{closure} of \( A \) is the intersection of closed sets containing \( A \).
Thus, we have
\[ \Int(A) \subset A \subset \Cl(A). \]
\end{definition}

\begin{example}
We have
\[ \Int \left ( \set*{\frac 1 n}_{n \in \N} \cup (1, 2] \right ) = (1, 2) \]
and
\[ \Cl \left ( \set*{\frac 1 n}_{n \in \N} \cup (1, 2] \right ) = \set 0 \cup \set*{\frac 1 n}_{n \in \N} \cup [1, 2]. \]
\end{example}

\begin{example}
Consider the following.
\wideimage{closureInterior.png}
Let \( A \subset \R^2 \) denote the set in the middle.
Then, the set on the left is the interior of \( A \) and the set of the right is the closure of \( A \).
\end{example}

\begin{remark}
Let \( A \subset Y \subset X \) be topological spaces with the subpace topology on \( Y \).
Then, there is a possibility that
\[ \Cl_Y(A) \subsetneq \Cl_X(A). \]
Indeed, if \( A = Y \), then \( \Cl_Y(Y) = Y \), whereas \( \Cl_X(Y) \) may include points not in \( Y \).
\end{remark}

\begin{theorem}
Let \( X \) be a topological space and \( Y \subset X \) be given the subspace topology.
Then, for all \( A \subset Y \) we have
\[ \Cl_Y(A) = \Cl_X(A) \cap Y. \]
\end{theorem}

\begin{theorem}
Let \( A \subset X \) be a subset of a topological space.
Then, \( x \in \Cl(A) \) if and only if every neighborhood of \( x \) intersects \( A \).
\end{theorem}

\begin{proof}
We prove by contraposition.
We show that if \( x \not \in A \) then there is some neighborhood of \( x \) which is disjoint from \( A \).
Indeed, if \( x \not \in A \) then \( x \in X - \Cl(A) \) is open and disjoint from \( A \).
On the other hand, we show that if there is some neighborhood that is disjoint from \( A \), then \( x \) is not in the closure of \( A \).
Indeed, if \( U \) is a neighborhood of \( x \) with \( A \cap U = \emptyset \), then \( A \) is contained in the closed subset \( X - U \). So, \( \Cl(A) \subset X - U \) and so \( x \not \in \Cl(A) \).
\end{proof}

\begin{definition}
A \emph{limit point} \( x \) of a subset \( A \subset X \) is a point for which every neighborhood of \( x \) intersects \( A - \set x \).
We denote the set of limits points of \( A \) by \( A' \).
\end{definition}

\begin{example}
Let \( A \) denote the following subset of \( \R^2 \).
\image{limitPoints.png}
Then, the set \( A' \) is the whole disc without the isolated point on the right.
\end{example}

\begin{example}
Take \( \Q \subset \R \).
Then, \( \Q' = \R \).
\end{example}

\begin{theorem}
Let \( X \) be a topological space.
Let \( A \subset X \).
Then, \( A \) is closed if and only if \( A' \subset A \).
\end{theorem}

\begin{example}
Consider the following finite topology.
\image{closureLimit.png}
Then, we have
\begin{align*}
\Cl(\set 1) &= \set{1, 2}, & \set 1' &= \set 2, \\
\Cl(\set 2) &= \set 2, & \set 2' &= \emptyset, \\
\Cl(\set 3) &= X, & \set 3' &= \set{1, 2}.
\end{align*}
\end{example}

\begin{example}
The trivial topology on \( X \) is not Hausdorff when \( \abs X > 1 \).
\end{example}

\begin{definition}
Let \( X \) be a topological space.
Then, \( X \) is said to be \( T_1 \) is singletons are closed.
\end{definition}

\begin{lemma}
Let \( X \) be a topological space.
If \( X \) is Hausdorff then \( X \) is \( T_1 \).
\end{lemma}

\begin{proof}
For \( p \neq q \), there exist disjoint neighborhoods \( U_p, U_q \) so
\[ \set p = \bigcap_{q \neq p} (X - U_q). \]
\end{proof}

\begin{corollary}
If \( X \) is a Hausdorff topological space, then finite subset are closed.
\end{corollary}

\begin{corollary}
If \( X \) is a finite Hausdorff topological space, then \( X \) has the discrete topology.
\end{corollary}

\begin{note}
Hausdorff and \( T_1 \) are the same for finite topological space.
They both result in the topology being discrete.
\end{note}

\begin{example}
The following finite topological spaces are not Hausdorff.
\wideimage{nonHausdorff.png}
\end{example}

\begin{example}[Line with two origins]
Consider the real line with two origins.
\image{lineWithTwoOrigins.png}
The space is given by
\[ X = (-\infty, 0) \sqcup \set{0_1, 0_2} \sqcup (0, +\infty) \]
with basis
\begin{align*}
\B &= \set{(a, b) \given a < b < 0 \tor 0 < a < b} \\
&\qquad \cup \set{(a, 0) \cup \set{0_1} \cup (0, b) \given a < 0 < b} \\
&\qquad \cup \set{(a, 0) \cup \set{0_2} \cup (0, b) \given a < 0 < b}
\end{align*}
\end{example}

\begin{theorem}
The order topology is Hausdorff.
\end{theorem}

\begin{proof}
We prove by cases.
If there is some \( s \in X \) such that \( p < s < q \) then \( p \in (-\infty, s) \) and \( q \in (s, +\infty) \) are disjoint neighborhoods of \( p, q \).
Otherwise, \( p \in (-\infty, q) \) and \( q \in (p, +\infty) \) are disjoint neighborhoods of \( p, q \).
\end{proof}

\begin{theorem}
If \( X \) is Hausdorff and \( Y \subset X \) is a subspace, then \( Y \) is Hausdorff.
\end{theorem}

\begin{proof}
If \( p \neq q \in Y \) then there exists disjoint neighborhoods \( U_p, U_q \) in \( X \) containing \( p \) and \( q \) respectively.
Then, we have disjoing neighborhoods \( U_p \cap Y \) and \( U_q \cap Y \) in \( Y \).
\end{proof}

\begin{theorem}
If \( A, B \) is Hausdorff, then \( A \cross B \) is Hausdorff.
\end{theorem}

\begin{proof}
Let \( (a_1, b-1) \neq (a_2, b_2) \).
We prove by cases.
If \( a_1 \neq a_2 \) then we have \( U_1, U_2 \) in \( A \) which separate \( a_1 \) and \( a_2 \).
Then \( U_1 \cross B \) and \( U_2 \cross B \) separate \( (a_1, b_1) \) and \( (a_2, b_2) \).
If \( b_1 \neq b_2 \) then we have \( V_1, V_2 \) in \( B \) which separate \( b_1 \) and \( b_2 \).
Then, \( A \cross V_1 \) and \( A \cross V_2 \) separate \( (a_1, b_1) \) and \( (a_2, b_2) \).
\end{proof}

\begin{definition}
The \emph{Finite Complement Topology} (FTC) on a set \( X \) has that a set \( U \) is open if and only if \( \card{X - U} < \infty \).
\end{definition}

\begin{remark}
The FCT is generated by the subbasis
\[ \mathcal S = \set{X -\set x}_{x \in X}. \]
Thus, FCT is the coarsest \( T_1 \) topology on \( X \).
\end{remark}

\begin{example}
If \( \card X < \infty \), then the FCT is discrete.
If \( \card X = \infty \), then the FCT is \( T_1 \) but not Hausdorff, since when \( \card{F_1} < \infty \) and \( \card{F_2} < \infty \), we have
\[ ( X - F_1) \cap (X - F_2) = (X - (F_1 \cup F_2)) \neq \emptyset. \]
\end{example}

\begin{definition}
A sequence \( (x_i)_{i \in \N} \) in \( X \) is said to \emph{converge to a point \( p \in X \)} if for each neighborhood \( U \) of \( p \), there is some \( N_U \in \N \) such that \( x_i \in U \) whenever \( i \geq N_U \).
\end{definition}

\begin{example}
Consider the following finite topological space.
\image{finiteSpaceSequence.png}
Then, the sequence
\[ a, b, a, b, a, b, \ldots \]
converges to \( a \), \( b \), \( c \) and \( d \), but not to \( e \).
\end{example}

\begin{example}
Let \( X \) be a topological space with the FCT.
Let \( (x_i)_{i \in \N} \) be a sequence such that \( x_i \neq x_j \) for \( i \neq j \).
Then for each \( p \in X \), the sequence \( (x_i)_{i \in \N} \) converges to \( p \).
To see why, consider a neighbohood of \( p \), \( X - F \) for some \( F \subset X \) with \( \card F < \infty \).
For \( f \in F \), define
\[ N_F = \begin{cases} 0 &\tif** f \not \in \set{x_i}_{i\in\N} \\ m &\tif** f = x_m. \end{cases} \]
Let
\[ N_F = \max_{f \in F} N_f + 1. \]
Then, \( x_i \not \in F \) for \( i \geq N_f \) so \( x_i \in X - F \) for all \( i \geq N_F \).
\end{example}
\end{document}