\documentclass[12pt]{book}
\usepackage[utf8]{inputenc}

\usepackage{import}

\import{./template/}{defaults}
\import{./template/}{common}
\import{./template/}{notes}


\title{Graph Theory}
\author{Diego Lopez and Gabriela Moisescu-Pareja}
\date{Fall 2020}

\begin{document}

\maketitle

\chapter{Definitions}

\begin{definition}
A \emph{graph} \( G \) is a pair of sets
\begin{enumerate}
\item a \emph{vertex set} \( V(G) \) consisting of vertices; and
\item an \emph{edge set} \( E(G) \) consisting of edges such that each edge has one or two vertices at its ends.
\end{enumerate}
\end{definition}

\begin{definition}
An edge is said to \emph{join} its ends.
\end{definition}

\begin{definition}
An edge with only one vertex is called a \emph{loop}.
\end{definition}

\begin{definition}
Two edges joining the same vertices are said to be \emph{parallel}.
\end{definition}

\begin{definition}
Two vertices are said to be \emph{adjacent} if they are joined by an edge.
\end{definition}

\begin{proposition}
Consider the following graph.
\image{bridges.png}
Then, there does not exist a path on this graph which traverses every edge exactly once.
\end{proposition}

\begin{proof}
Assume for the purpose of contradiction that were was a path.
Then this path would have a starting vertex and an ending vertex.
Consider the two vertices which are neither start nor end vertices.
Then, since these neither start nor end the path, each time that this vertex is visited it must also be exited from a different edge.
This means that there would be an even number of edges containing this vertex.
However, in the case of our graph, all edges have an odd number of vertices, and so this cannot happen.
\end{proof}

\begin{proposition}
In a graph with 6 nodes, there are either three pairwise adjacent vertices or three pairwise non-adjacent vertices.
\end{proposition}

\begin{proof}[Sketch]
Number the edges 1 through 6.
This can be shown by a case analysis on how many edges contain the edge 1.
That is, there are either at least three vertices adjacent to 1 or three vertices non-adjacent to 1.
Then analysing each case can show that no matter what, there will always be either three edges which are pairwise adjacent or three edges which are pairwise non-adjacent.
\end{proof}

\begin{definition}
Let \( R(k) \) denote the minimum integer such that in every graph with \( \card{V(G)} = R(k) \), there are either \( k \) pairwise adjacent vertices or \( k \) pairwise non-adjacent vertices.
\end{definition}

\begin{note}
We showed that \( R(3) \leq 6 \).
We can show that \( R(3) = 6 \) by constructing a graph with 5 vertices which does not contain 3 pairwise adjacent vertices nor 3 pairwise non-adjacent vertices.
\end{note}

\begin{definition}
The \emph{null graph} has no vertices and no edges.
\end{definition}

\begin{note}
What is the maximum number of edges in a graph with \( n \) vertices?
It is infinity, as we can have an arbitrary number of parallel edges.
What about simple graphs?
In this case, there are \( n \) choices for the first end, and \( n-1 \) choices for the second end.
However, every edge is counted twice, we te have the the total number of edges is
\[ \frac{n (n-1)}2 = \binom n 2. \]
\end{note}

\begin{definition}
The simple graph with every pair of vertices adjacent is called \emph{complete}. The complete graph with \( n \) vertices is called \( K_n \).
\end{definition}
\end{document}