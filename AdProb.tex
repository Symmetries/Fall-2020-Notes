\documentclass[12pt]{book}
\usepackage[utf8]{inputenc}

\usepackage{import}

\import{./template/}{defaults}
\import{./template/}{common}
\import{./template/}{notes}


\renewcommand{\F}{\mathcal F}
\renewcommand{\P}{\mathbb P}

\title{Probability Theory}
\author{Diego Lopez and Gabriela Moisescu-Pareja}
\date{Fall 2020}

\begin{document}

\maketitle

\chapter{Introduction}

\section{General Measure Spaces}
\begin{definition}
Let \( S \) be a set.
A collection \( \Sigma_0 \) of subsets of \( S \) is called an \emph{algebra} if
\begin{enumerate}
\item \( S \in \Sigma_0 \)
\item if \( A \in \Sigma_0 \), then \( A\complement \coloneqq S \setminus A \in \Sigma_0 \);
\item for every \( n \geq 1 \), if \( A_1, A_2, \ldots, A_n \in \Sigma_0 \) then \[ \bigsqcup_{i=1}^n A_i \in \Sigma_0; \]
\item \( \emptyset \in \Sigma_0 \);
\item if \( A, B \in \Sigma_0 \) then \( A \cap B \), \( A \cup B \), \( A \setminus B \) and \( A \mathbin{\triangle} B \) are all in \( \Sigma_0 \);
\item if \( A_1, \ldots, A_n \in \Sigma_0 \) then \[ \bigcap_{i=1}^n A_i \in \Sigma_0. \]
\end{enumerate}
\end{definition}

\begin{definition}
A collection \( \Sigma \) of subsets of a set \( \Omega \) is a \emph{\( \sigma \)-algebra} if and only if \( \Sigma \) is an algebra and moreover
\begin{enumerate}
\item[3'.] if \( A_1, A_2, \ldots \in \Sigma \) then \[ \bigsqcup_{i=1}^\infty A_n \in \Sigma, \]
\end{enumerate}
that is, \( \Sigma \) is complete under countable unions.
\end{definition}

\begin{proposition}
If \( \Sigma \) is a \( \sigma \)-algebra over a set \( \Omega \), then we have
\begin{enumerate}
\item[6'.] if \( A_i \in \Sigma \) for all \( i \in \N \) then \[ \bigcap_{i=1}^\infty A_i \in \Sigma. \]
\end{enumerate}
\end{proposition}

\begin{definition}
A pair \( (S, \Sigma) \) of a set \( S \) together with a \( \sigma \)-algebra on that set is called a \emph{measurable space}.
A set \( A \in \Sigma \) is called a \emph{measurable set}.
\end{definition}

\begin{definition}
Let \( \mathcal C \) be a collection of subsets of \( S \).
The \emph{\( \sigma \)-algebra generated by \( \mathcal C \)}, denoted by \( \sigma(\mathcal C) \) is the smallest \( \sigma \)-algebra that contains \( \mathcal C \).
That is, if \( \Sigma' \) is a \( \sigma \)-algebra that contains \( \mathcal C \), then \( \sigma(\mathcal C) \subset \Sigma' \).
\end{definition}

\begin{note} \skipline
\begin{itemize}
\item If \( \mathcal C \) is a \( \sigma \)-algebra, then \( \sigma(\mathcal C) = \mathcal C \).
\item We always have \( \sigma(\sigma(\mathcal C)) = \sigma(\mathcal C) \).
\item If \( \mathcal C_1 \subset \mathcal C_2 \), then \( \sigma(\mathcal C_1) \subset \sigma(\mathcal C_2) \).
\end{itemize}
\end{note}

\begin{proposition}
If \( \set{\Sigma_\alpha : \alpha \in I} \) is any family of \( \sigma \)-algebra on \( S \), then
\[ \bigcap_{\alpha \in I} \Sigma_\alpha \]
is also a \( \sigma \)-algebra.
\end{proposition}

\begin{proposition}
Let \( \mathcal C \) be any family of subsets of a set \( S \).
Then, we have
\[ \sigma(\mathcal C) = \bigcap_{\Sigma \supset \mathcal C} \Sigma. \]
\end{proposition}

\begin{proof}
Clearly, we have that
\[ \mathcal C \subset \bigcap_{\Sigma \supseteq \mathcal C} \Sigma \]
and so we have
\[ \sigma(\mathcal C) \subset \bigcap_{\Sigma \supseteq \mathcal C} \Sigma. \]
For the other inclusion, we have that \( \sigma(\mathcal C) \supseteq \mathcal C \) and so we have
\[ \bigcap_{\Sigma \supset \mathcal C} \Sigma \subset \sigma(\mathcal C). \]
\end{proof}

\begin{exercise} \skipline
\begin{enumerate}
\item If \( \mathcal C = \set A \), \( A \subset S \), then what is \( \sigma(\mathcal C) \)?
\item If \( \mathcal C = \set{A, B} \), \( A, B \subset S \), then what is \( \sigma(\mathcal C) \)?
\end{enumerate}
\end{exercise}





\section{Probability Spaces}

\begin{definition}
A \emph{probability space} is a triple \( (\Omega, \F, \P) \) where
\begin{enumerate}
\item \( \Omega \) is a set called the sample space, informally a set of possible outcomes;
\item \( \F \) is a \( \sigma \)-algebra on \( \Omega \), informally the set of all events of special interest, that is, \( \F \subset \mathcal P(\Omega) \).
\item \( \P \) is a probability measure, that is it is a set function \( \P : \F \to [0, 1] \) which is a measure.
\end{enumerate}
\end{definition}

\begin{example}
Consider the experiment consisting of flipping a coin infinitely many times.
In this case, the sample space is \( \Omega = \set{H, T}^\N \).
That is, an event \( w \in \Omega \) is a sequence
\[ \omega = (\omega_1, \omega_2, \omega_3, \ldots) \]
where \( \omega_i \in \set{H, T} \) for all \( i \in \N \).
For any \( k \geq 1 \), we consider increasing finite sequences
\[ \vec n_k \coloneqq (n_1, n_2, \ldots, n_k) \in \N^k, \quad n_1 < n_2 < \ldots < n_k \]
and an associated finite sequence
\[ \vec \xi_k \coloneqq (\xi_1, \xi_2, \ldots, \xi_k) \in \set{H, T}^k. \]
Given \( \vec n_k \) and \( \vec \xi_k \), we define
\[ A_{\vec n_k, \vec \xi_k} \coloneqq \set{\omega \in \Omega \given \omega_{n_j} = \xi_j,\ \forall 1 \leq j \leq k}. \]
Then, we take \( \F \) to be the \( \sigma \)-algebra generated by the collection
\[ \set*{A_{\vec n_k, \vec \xi} \given k \in \N,\ \vec n_k,\ \vec \xi_k}. \]
In this case, we have
\[ \P(A_{\vec n_k, \vec \xi_k}) = \frac 1 {2^k}. \]
\end{example}
\end{document}